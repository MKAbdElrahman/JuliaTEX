\documentclass[a4paper,11pt]{scrartcl}
\usepackage{minted}
\usepackage{hyperref}
\begin{document}
	\title{How to typeset  Julia in LATEX source files }
	\date{}
	\author{M.K AbdElrahman}
	\maketitle

\newcommand{\jlinline}[1]{\mintinline{jl.py:Julia1Lexer -x}{#1}}

\newminted[jl]{jl.py:Julia1Lexer -x}{frame=lines,framerule=1pt,linenos,fontfamily=courier,framesep=2mm,fontsize=\scriptsize,xleftmargin=21pt}

\section{Introduction}
I've been searching for how typeset  Julia in TEX files, the problem is package  minted doesn't work for custom defined Julia types. I found this lexer solution on the Web.  
\section{Steps}	
Instead of installing the lexer in your Pygments installation, you can keep the lexer with your TeX documents.	
\begin{enumerate}
	 \item  If your lexer file is named "jl.py" and the class is "Julia1Lexer", put jl.py in the same directory as your main TeX source file
	 \item Specify the language in minted with \textbf{ \\begin\{minted\}\{jl.py:Julia1Lexer -x\}  \#code
	 	\\end\{minted\}.}
\end{enumerate}
\section{Inline code Example}
This is an inline code \jlinline{f(x::Array) = sin.(x)}
\section{Listing Example}
\begin{jl}
	abstract type IntegrationMethod end
	struct Trapezoidal <: IntegrationMethod end
	"""
	integrate(x::AbstractVector, y::AbstractVector, ::Trapezoidal)
	Use Trapezoidal rule.
	"""
	function integrate(x::AbstractVector, y::AbstractVector, ::Trapezoidal)
		h = (x[2] - x[1])
		I = h * (y[1] / 2 + sum(y[2:end-1]) + y[end] / 2)
		return I
	end
	
\end{jl}

\section{References}
\begin{itemize}
	\item The solution steps  were suggested by 
	\href{https://tex.stackexchange.com/users/11030/jonathan-schuster}{Jonathan Schuster} in answer to  \href{https://tex.stackexchange.com/questions/18083/how-to-add-custom-c-keywords-to-be-recognized-by-minted#comment930474_42392}{How to add custom C++ keywords to be recognized by Minted?}
	\item  The custom lexer used here is a based on a laxer  modified by  Mykel J. Kochenderfer. \href{https://mykel.kochenderfer.com/}{Mykel J. Kochenderfer} and is available online  at \href{https://github.com/sisl/pygments-julia/tree/f2d897ceb5d66fa83fb1870ac047ff19987acc88}{Pygments Julia Lexer}
\end{itemize}


\end{document}