\documentclass[a4paper,12pt]{scrartcl}
%%%%%%%%%%%%%%%%%%%
%for code highlighting
\usepackage{minted} 
%for supporting unicode characters
\usepackage[mathletters]{ucs} \usepackage[utf8x]{inputenc}
%%%%%%%%%%%%%%%%%%%
%%%%%%%%%%%%%%%%%%%
%The infitisimal d#
\newcommand \diff[1] { \,\textrm d{#1}} 
%bold vector
\newcommand{\vect}[1] {\boldsymbol{#1}}
%%%%%%%%%%%%%%%%%%%%
%%%%%%%%%%%%%%%%%%%%
%%%%%%%%%%%%%%%%%%%%
\begin{document}
	\title{Adding Julia Code in Latex \\
	Example}
	\maketitle
	\section{One Dimensional Integration}
	\begin{equation}
		\int_{a}^{b} f(x) \diff{x} \approx (b-a) f\left( \frac{b+a}{2}\right) 
	\end{equation}

\begin{listing}[H]
	\begin{minted}
	[frame=lines,
	framesep=2mm,
	baselinestretch=1.2,
	fontsize=\small,
	linenos]
	{julia}
	function integrate(f,a::Real,b::Real)
		return (b-a)*f((a+b)/2.0)
	end
	\end{minted}
	 \caption{Example of a Julia code in \LaTeX.}
	\label{lst:example}
\end{listing}

\end{document}